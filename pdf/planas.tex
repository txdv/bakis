\documentclass{VUMIFPSbakalaurinis}
\usepackage{algorithmicx}
\usepackage{algorithm}
\usepackage{algpseudocode}
\usepackage{amsfonts}
\usepackage{amsmath}
\usepackage{bm}
\usepackage{caption}
\usepackage{color}
\usepackage{float}
\usepackage{graphicx}
\usepackage{listings}
\usepackage{subfig}
\usepackage{wrapfig}

% Titulinio aprašas
\university{Vilniaus universitetas}
\faculty{Matematikos ir informatikos fakultetas}
\institute{Informatikos institutas}  % Užkomentavus šią eilutę - institutas neįtraukiamas į titulinį
\department{Programų sistemų bakalauro studijų programa}
\papertype{Bakalauro baigiamojo darbo planas}
\title{Kompiliatoriaus transliuojamo vieneto smulkinimas}
\titleineng{Translation unit granularization}
\author{Andrius Bentkus}
% \secondauthor{Vardonis Pavardonis}   % Pridėti antrą autorių
\supervisor{asist. dr. Vytauts Valaitis}
\reviewer{}
\date{Vilnius – \the\year}

% Nustatymai
% \setmainfont{Palemonas}   % Pakeisti teksto šriftą į Palemonas (turi būti įdiegtas sistemoje)
\bibliography{bibliografija}

\begin{document}
\maketitle

%% Padėkų skyrius
% \sectionnonumnocontent{}
% \vspace{7cm}
% \begin{center}
%     Padėkos asmenims ir/ar organizacijoms
% \end{center}

\begin{samepage}
%Planas
\section{Plan}
%Tyrimo objektas ir aktualumas
\subsection{Research object and actuality}
The majority of the research done on compilers is focused on making compilers output optimized programs in terms of code execution speed and memory usage \cite{lopes2018future} while neglecting or wilfully sacrificing \cite{fast2019compilers} actual runtime performance of the compiler itself.
As computers get faster more resources are available, but programmers tend to utilize these newfound resources to make the implementation of programs simpler rather than making the programs faster.

Another vector of academic improvement within the compiler sphere is to add new features or utilize new paradigms like dependent types or formal specifications in order to allow the compiler to do more sophisticated type checking and verification than classic type checking allows while significantly increasing runtime compilation.

When assessing the adoption of a new programming language the runtime speed of the available compilers for a particular programming languages are often gauged \cite{ScalaSlow} as an important metric among other such as language complexicity, available libraries and spread of ecosystem.
Dealing with enormous compilation times of huge projects can have negative effectives on the producitivity of programmers and even drive away people from an entire ecosystem \cite{ScalaReallySlow}.
%Look at this example where  people refuse to adopt languages because they compilation times is really bad.

Fast compilers are a thing now, because go is so awesome.


% Keliami uždaviniai ir laukiami rezultatai
\subsection{Challenges and expected results}
The goal of this bachelor thesis is to create a minimal compiler for a small subset of the Scala programming language.
Then an attempt will be made to introduce translation unit granularization.
A translation unit is usually the entire input of source code that is compiled to a compiler it usually gets recompiled in its entirity even though most of the input does not changes inbetween compilation runs.

Scala is known to have a multitude of advanced and complicated features\cite{ScalaSpec}, however the targeted compiler implementation in this thesis will support only the most basic features in order to allow for an easier approach of implementation.

% Tyrimo Metodas
\subsection{Investigation method}
% Darbo atlikimo procesas
\subsection{Work process}
% Used literature
\subsection{Used literature}
\end{samepage}

\printbibliography[heading=bibintoc]  % Šaltinių sąraše nurodoma panaudota
% literatūra, kitokie šaltiniai. Abėcėlės tvarka išdėstomi darbe panaudotų
% (cituotų, perfrazuotų ar bent paminėtų) mokslo leidinių, kitokių publikacijų
% bibliografiniai aprašai. Šaltinių sąrašas spausdinamas iš naujo puslapio.
% Aprašai pateikiami netransliteruoti. Šaltinių sąraše negali būti tokių
% šaltinių, kurie nebuvo paminėti tekste. Šaltinių sąraše rekomenduojame
% necituoti savo kursinio darbo, nes tai nėra oficialus literatūros šaltinis.
% Jei tokių nuorodų reikia, pateikti jas tekste.

% \sectionnonum{Sąvokų apibrėžimai}

\appendix  % Priedai
% Prieduose gali būti pateikiama pagalbinė, ypač darbo autoriaus savarankiškai
% parengta, medžiaga. Savarankiški priedai gali būti pateikiami ir
% kompaktiniame diske. Priedai taip pat numeruojami ir vadinami. Darbo tekstas
% su priedais susiejamas nuorodomis.

\end{document}
